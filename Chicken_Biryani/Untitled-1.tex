\documentclass{recipe}
\usepackage{hyperref}
\author{Jyothi Rajesh}
\title{Chicken Dum Biryani Recipe}
\date{6\textsuperscript{th} June 2016}
\begin{document}
    \maketitle   
    \section{Introduction}
        \subsection{Desciption}
            An aromatic and flavourful, classic Chicken Dum Biryani Recipe, is a favourite recipe amongst all Indian households. Serve it with pickled onions and raita for a hearty Sunday lunch.   
        Chicken Dum Biryani Recipe is one such recipe which is loved by everyone in the family.
        \subsection{History}
            Biryani is derived from the Farsi word ‘birian’ originated in Persia, biryani was introduced to India during the British rule.
        \subsection{Origin}
            Biryani preparation varies every 100 miles within India. Each of the variation being loyal to its regional history. Down south in most regions like Karnataka and Tamil Nadu, biryani is usually made using small grain rice called “jeera samba rice”. But in Northern parts of India biryani is compulsorily made using long grain rice called “basmati”. No matter what region it is, or style, any biryani tastes simply delicious. 
        \subsection{Serving Suggestions}
            Serve Chicken Dum Biryani Recipe with Pickled onions, Burani Raita and Mirchi Ka Salan Recipe (Chillies in Tangy Spicy Peanut Sesame Curry) for Sunday afternoon lunch. 
    \section{Recipe}
        \subsection{Equipment}
            \begin{itemize}
                \item Cuisine:  North Indian
                \item Course:   One Pot Dish
                \item Diet:     High Protien Non Vegetarian
                \item Equipment Used:   	Large Cooking Pot, Cast Iron Cooking Pot/ Casserole
            \end{itemize}
        \subsection{Portion Size and Time}
            \begin{itemize}
                \item Preperation Time: 30 minutes
                \item Cooking Time: 2 hours
                \item Serves:   6-8 people
            \end{itemize}
        \subsection{Ingredients}
            \begin{itemize}
                \item 1 kg Chicken, with bone 
                \item 4 cups Basmati rice
                \item 3 Onions, sliced thin 
                \item 4 Tomatoes, finely chopped
                \item 4 Green Chillies, slit at the center
                \item 20 gram Ginger, ground to paste
                \item 20 gram Garlic, ground to paste
                \item 2 tablespoon Red Chilli powder	
                \item 2 tablespoon Coriander Powder (Dhania)	
                \item 1 tablespoon Garam masala powder	
                \item 1/2 cup Coriander (Dhania) Leaves, finely chopped
                \item 1/2 cup Mint Leaves (Pudina), finely chopped
                \item Oil, as needed
                \item Ghee, as needed
                \item Salt, to taste 
            \end{itemize}
            \textbf{For marinating} 
            \begin{itemize}
                \item 1/2 cup Hung Curd (Greek Yogurt)	
                \item 1 tablespoon Red Chilli powder	
                \item 1 teaspoon Turmeric powder (Haldi)	
                \item Salt, as required 
            \end{itemize}   
        \subsection{Instructions}
        \begin{enumerate}
            \item To begin making the Chicken Biryani recipe, we will first begin with the rice preparation. To cook basmati rice separately, wash and soak the basmati rice in water for 20 minutes.

            \item Meanwhile add about 3 litres of water in a large vessel and bring it to boil.
            
            \item Once water starts to boil, add about 1 teaspoon of cooking oil and 1 teaspoon of salt. Add the soaked and drained rice, stir gently once and cook for about 3 to 5 minutes (I usually take it off heat by 3-1/2 minutes).
            
            \item Keep an eye on rice as some brands of basmati rice cooks very fast and some takes time. My rice was cooked in 5 minutes. Drain the water immediately and spread the rice on a large plate.
            
            \item In a deep bottomed pan, heat oil on medium flame,  add 1 sliced onion and fry it until brown (don’t burn the onions). Remove browned onion from oil and keep it aside. This will be used for garnishing and to make the layers.
            
            \item The next step is to marinate the chicken for the Chicken Dum Biryani Recipe.
            
            \item Clean and wash the chicken thoroughly.
            
            \item In a mixing bowl, marinate the chicken with thick curd, red chili powder, turmeric powder and salt.
            
            \item Marinate for at least 30 minutes. 
            
            \item In a large pan, add oil and heat on medium flame, add the remaining sliced onions and saute for 3 minutes or until the onions turn translucent. Next add ginger-garlic paste and saute till the raw smell goes off.
            
            \item Add slit green chilies and mix for a minute.
            
            \item To this add chopped tomatoes and saute till tomatoes turn slightly mushy.
            
            \item Once the tomatoes are mushy add red chili powder, coriander powder and salt and saute it for about 2 minutes.
            
            \item Add 3 tablespoons of both coriander and mint leaves and mix.
            
            \item Finally, add  the marinated chicken for the biryani and mix well. Cook until chicken is fully cooked. 
            
            \item Once chicken is cooked, if you find there is excess water, increase the flame and thicken the masala. We need the masala just to coat the chicken.
            
            \item Take a large wide, deep vessel, add ghee and spread it to coat all the bottom and sides of the pan, reduce the flame to low. Add about 2 full tablespoons of the thick chicken biryani masala and spread it all over the bottom.
            
            \item Next, add cooked basmati rice over the chicken biryani masala and gently spread the rice to cover the chicken. 
            
            \item Use a tea spoon and remove the top oily layer from masala and drizzle over the rice, this will add flavours to the rice and also give colour.
            
            \item Repeat the layering process until you have used up all rice and chicken. Over the top sprinkle coriander and mint leaves and browned onions and close with a lid.
            
            \item Ensure the flame is low and place a heavy weight on top of lid to trap the steam inside to cook the Chicken Dum Biryani recipe. Leave the dum biryani aside for 10 minutes until all the flavors seep through.
            
            \item Scoop out the Chicken Dum Biryani from the edges of the pan, making sure not to break the rice grains.
            
            \item Serve Chicken Dum Biryani Recipe with Pickled onions, Burani Raita and Mirchi Ka Salan Recipe (Chillies in Tangy Spicy Peanut Sesame Curry) for Sunday afternoon lunch.
        \end{enumerate}
   
       
    \pagenumbering{gobble}
\end{document}
