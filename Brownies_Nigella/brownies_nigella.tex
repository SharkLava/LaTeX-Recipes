\documentclass{article}
\usepackage{hyperref}
\usepackage[margin=1in]{geometry}
\author{Nigella Lawson}
\title{Brownies}
\begin{document}
    \maketitle   
    \section{Introduction}
        I don't know why people don't make Birthday Brownies all the time - they're so easy and so wonderful. Brownies are much quicker to make than a cake, and they look so beautiful piled up in a rough-and-tumble pyramid spiked with birthday candles.
    \section{Recipe}
        \subsection{Ingredients}    
        \begin{itemize}        
           \item 375 grams soft unsalted butter
           \item  375 grams best-quality dark chocolate
           \item 6 large eggs
           \item 1 tablespoon vanilla extract
           \item 500 grams caster sugar
           \item 225 grams plain flour
           \item 1 teaspoon salt
           \item 300 grams chopped walnuts
        \end{itemize}
        \subsection{Instructions}
            You will need a tin measuring approximately 33 x 23 x 5$\frac{1}{2}$ / 13 x 9 x 2$\frac{1}{4}$ inches
            \begin{enumerate}            
                \item Preheat the oven to 180°C/160°C Fan/350°F/gas mark 4. Line your brownie pan - I think it's worth lining the sides as well as the base - with foil, parchment or Bake-O-Glide. 
                \item Melt the butter and chocolate together in a large heavy-based pan. In a bowl or large wide-mouthed measuring jug, beat the eggs with the sugar and vanilla. Measure the flour into another bowl and add the salt. 
                \item When the chocolate mixture has melted, let it cool a bit before beating in the eggs and sugar, and then the nuts and flour. Beat to combine smoothly and then scrape out of the saucepan into the lined pan. 
                \item Bake for about 25 minutes. When it's ready, the top should be dried to a paler brown speckle, but the middle still dark and dense and gooey. And even with such a big batch you do need to keep alert, keep checking: the difference between gungy brownies and dry brownies is only a few minutes; remember that they will continue to cook as they cool.
            \end{enumerate}           
    \section{Additional Information}
            {\bfseries \large Variations:   }
                You can really vary brownies as you wish: get rid of the walnuts, or halve them and make up their full weight with dried cherries; or replace them with other nuts – peanuts, brazils, hazelnuts – add shredded coconut or white chocolate chips or buttons; try stirring in some Jordan’s Original Crunchy cereal. I had high hopes for chic, after-dinner pistachio-studded brownies, but found the nuts get too soft and waxy, when what you need is a little crunchy contrast.
    \pagenumbering{gobble}
\end{document}
 
