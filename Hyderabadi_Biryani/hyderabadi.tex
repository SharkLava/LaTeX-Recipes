\documentclass{article}
\usepackage{hyperref}
\usepackage[margin=1in]{geometry}
\author{Sanjeev Kapoor}
\title{Hyderabadi Biryani}
\begin{document}
    \maketitle   
    \section{Introduction}
        When it comes to biryanis this stalwart is the first one that pops up in our mind. Succulent mutton chunks – check. Fragrant long grain basmati rice – check, dum cooked – check. This true blue, tried and tested authentic kachchi biryani recipe from Hyderabad has satisfied even the most stringent critics. If you haven’t tried it one yet, your biryani story is incomplete

        This recipe has featured on the show Khanakhazana.

        A biryani from Hyderabad is like its people, exciting, spicy, sweet and tangy – all in one!  Pieces of marinated mutton, long grain basmati rice and fragrant whole spices layered and cooked together on low heat till the meat is tender and falls of the bone and the rice has fluffed up and is soft.  Fried onions or barista, kewra water and saffron strands added to this brings along plenty more aroma and flavour! Sunday brunch or house party – this biryani is fit for any occasion!
    \section{Recipe}
        \subsection{Portion Size and Time}
            \begin{itemize}
                \item Prep Time : 2-2.30 hour
                \item Cook time : 41-50 minutes
                \item  Serve : 4
                \item Level Of Cooking : Moderate
                \item Taste : Mild
            \end{itemize}
        \subsection{Ingredients}
            \begin{itemize}
                \item Mutton a mix of chops, marrowbone and shoulder pieces 500 grams
                \item Basmati rice 1 1/2 cups
                \item Salt to taste
                \item Bay leaves 2
                \item Green cardamoms 10
                \item Black peppercorns 25-30
                \item Cinnamon 3 inch sticks
                \item Oil 1 tablespoon + to deep fry
                \item Onions sliced 5 large
                \item Caraway seeds (shahi jeera) 1/2 teaspoon
                \item Cloves 10
                \item Ginger paste 1 tablespoon
                \item Garlic paste 1 tablespoon
                \item Red chilli powder 1 tablespoon
                \item Yogurt 1 cup
                \item Fresh coriander leaves torn 2 tablespoons
                \item Fresh mint leaves torn 2 tablespoons
                \item Pure ghee 4 tablespoons
                \item Black cardamoms 2
                \item Saffron (kesar) mix in 1/4 cup milk a few strands 
            \end{itemize}
              
        \subsection{Instructions}
        \begin{enumerate}
            \item Heat five to six cups of water in a deep pan. Add drained rice, salt, bay leaves, five green cardamoms, seven to eight black peppercorns, one cinnamon stick and cook till three fourth done.
            \item Drain and set aside. Heat sufficient oil in a kadai and deep-fry half the onion slices till golden. Drain and place on an absorbent paper. Grind caraway seeds, one cinnamon stick, remaining black peppercorns, cloves and remaining green cardamoms to a fine powder and set aside.
            \item Take mutton pieces in a bowl. Add ginger paste, garlic paste and salt and mix. Add the spice powder, red chilli powder, half the fried onions crushed, yogurt, coriander leaves, half of the mint leaves and one tablespoon oil and mix. Let it marinate for about two hours in the refrigerator. 
            \item Heat two tablespoons ghee in a pan, remaining cinnamon and black cardamoms and sauté till fragrant. Add remaining onions and sauté till light golden. Add marinated mutton, stir and cook on high heat for three to four minutes. Cover, reduce heat and cook till almost done. 
            \item Heat the remaining ghee in a thick-bottomed pan. Spread half the rice in a layer. Spread the mutton over the rice. Sprinkle remaining torn mint leaves. Spread the remaining rice. Sprinkle saffron milk. Cover and cook under dum till done. Serve hot with a raita of your choice.
        \end{enumerate}
    \section{Nutrition Info}
        \begin{itemize}
            \item Calories : 3388 Kcal
            \item Carbohydrates : 300.3 gm
            \item Protein : 129.7 gm
            \item Fat : 185.3 gm
            \item Other : 0 
        \end{itemize}
       
    \pagenumbering{gobble}
\end{document}
